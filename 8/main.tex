\documentclass[11pt, letterpaper]{article}
\usepackage[top=80pt,bottom=80pt,left=60pt,right=60pt]{geometry}
\usepackage{listings}
\usepackage{amsmath}
\usepackage{titling}
\newcommand{\subtitle}[1]{
  \posttitle{
    \par\end{center}
    \begin{center}\large#1\end{center}
    \vskip0.5em}
}

\begin{document}
  \title{CSCI 3104 Assignment 8}
    \subtitle{10:00 - 10:50 Wanshan}
    \date{April 11 2016}
    \author{Jackson Chen}
    \maketitle

  \begin{enumerate}
    % Question 1
    \item
      \begin{enumerate}
        \item
          Topological ordering: 1, 2, 3, 4, 5 \\
          Topological sorting takes $\Theta (m + n)$
        \item
          Pseudocode for topological shortest path finding:
          \begin{lstlisting}
            def findShortest(startNode):
              dist = [0, Infinity, Infinity, Infinity, ...]
              path = [nil, nil, ...]
              // length of path and dist is the # of vertices
              dist, path = find(startNode, dist, path)
              return dist, path

            def find(node, dist, path):
              if (node.children.exist?):
                for (node.children):
                  dist, path = relax(dist, path, node, weight)
                  find(node.child, dist, path)
              return (dist, path)
          \end{lstlisting}
          The algorithm first gives an infinite distances to all nodes except 2 (which has dist 0).
          Then the children of node 2 (nodes 3, 4, and 5) are assigned distances 2.4, 1.3, and -0.3 respectively.
          Then the edge between node 3 and its child (node 5) is relaxed, but that distance to 5 is not changed since 2.4 - 1 is still larger than -0.3.
          Then the edge between node 4 and its child (node 5) is relaxed, but that distance to 5 is not changed since 1.3 + 2 is still larger than -0.3.
          Thus the final distances for nodes 1-5, respectively, are infinity, 0, 2.4, 1.3, -0.3.
        \item
          $\Theta (n + m)$
      \end{enumerate}
    \item
      \begin{enumerate}
        \item
          Each edge can have a weight of 2. Then there are back edges from the last character of a known codeword to the first character,
          which would negate the sum of the weights that connected all the letters (but one less than that to allow codewords to contribute
          a cost of 1). Adding new edges isn't an issue since all edges are of weight 2 except for backedges. Thus if it's a codeword that
          is being added, then the backedge needs to be adjusted to reflect the addition of the new edge.
        \item
          Since codewords are strongly connected components with a net weight of 1 upon traversal, it incentives the algorithm to traverse
          those characters that make up the codeword once. Then it follows the path of ciphered characters as normal.
      \end{enumerate}
  \end{enumerate}
\end{document}
