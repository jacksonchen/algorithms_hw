\documentclass[11pt, letterpaper]{article}
\usepackage[top=80pt,bottom=80pt,left=60pt,right=60pt]{geometry}
\usepackage{listings}
\usepackage{amsmath}
\usepackage{titling}
\newcommand{\subtitle}[1]{
  \posttitle{
    \par\end{center}
    \begin{center}\large#1\end{center}
    \vskip0.5em}
}

\begin{document}
  \title{CSCI 3104 Assignment 6}
    \subtitle{10:00 - 10:50 Wanshan}
    \date{March 14 2016}
    \author{Jackson Chen}
    \maketitle

  \begin{enumerate}
    % Question 1
    \item
      \begin{enumerate}
        \item
          LIS(a, 3, 2) = 1 \\
          LIS(a, 5, infinity) = 3
        \item
          LIS(a, 0, M) = 0 \\
          LIS([], j, M) = 0
        \item
          \[ LIS(a, j, M) = \begin{cases}
                              1 + LIS(a, j-1, a[j]) & \text{if } a[j] < M \\
                              LIS(a, j-1, M) & \text{if } a[j] >= M
                            \end{cases} \]
        \item
          A bottom-up scheme would consist of a table that solved LIS(a, j, M).
          The columns would consist of arrays [], a[0], a[0, 1], a[0, 1, 2], a[0, 1, 2, .., j-1].
          All in all, there would be a total of j+1 columns. The rows would consist of the different
          values for M, from 0 to the maximum value in a + 1. The number of rows would be the max value in a + 2.
          The base cases would be 0. If the array to fill is an empty array, the longest increasing sequence
          is of length 0. If there is an array, but the numbers in the sequence cannot be any of the numbers in
          array (because they are bigger than M) then the longest sequence is also length 0. Each entry is filled out
          based on the upper and left values.
          Memo table:
          \begin{table}[h]
          \centering
          \begin{tabular}{lllllll}
                     & \textbf{{[}{]}} & \textbf{{[}1{]}} & \textbf{{[}1, 5{]}} & \textbf{{[}1, 5, 2{]}} & \textbf{{[}1, 5, 2, 3{]}} & \textbf{{[}1, 5, 2, 3, 8{]}} \\
          \textbf{0} & 0               & 0                & 0                   & 0                      & 0                         & 0                            \\
          \textbf{1} & 0               & 1                & 1                   & 1                      & 1                         & 1                            \\
          \textbf{2} & 0               & 1                & 1                   & 1                      & 1                         & 1                            \\
          \textbf{3} & 0               & 1                & 1                   & 2                      & 2                         & 2                            \\
          \textbf{4} & 0               & 1                & 1                   & 2                      & 3                         & 3                            \\
          \textbf{5} & 0               & 1                & 1                   & 2                      & 3                         & 3                            \\
          \textbf{6} & 0               & 1                & 2                   & 2                      & 3                         & 3                            \\
          \textbf{7} & 0               & 1                & 2                   & 2                      & 3                         & 3                            \\
          \textbf{8} & 0               & 1                & 2                   & 2                      & 3                         & 3                            \\
          \textbf{9} & 0               & 1                & 2                   & 2                      & 3                         & 4
          \end{tabular}
          \end{table}
      \end{enumerate}
      \item
        There are a total of 21 paths from node 0 to node 13. In order to achieve this number, sum the
        number of paths it takes to get to the adjoining nodes. For example, nodes 0, 1, 2, 3, 4, 5 all have
        1 possible path to reach them. To reach node 6, sum the two nodes that lead to it (0 and 1), so 1+1=2.
        Node 6 has 2 possible paths. Node 7 has 2+1+1=4. Node 8 4+1+1=6, and so on until node 13 is reached.
  \end{enumerate}
\end{document}
