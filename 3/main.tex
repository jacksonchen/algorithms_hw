\documentclass[11pt, letterpaper]{article}
\usepackage[top=80pt,bottom=80pt,left=60pt,right=60pt]{geometry}
\usepackage{listings}
\usepackage{amsmath}
\usepackage{titling}
\newcommand{\subtitle}[1]{
  \posttitle{
    \par\end{center}
    \begin{center}\large#1\end{center}
    \vskip0.5em}
}

\begin{document}
  \title{CSCI 3104 Assignment 3}
    \subtitle{10:00 - 10:50 Wanshan}
    \date{14 February 2016}
    \author{Jackson Chen}
    \maketitle

  \begin{enumerate}
    % Question 1
    \item
      \begin{lstlisting}
        def smallestR(a1, a2, r):
            smallestCounter = 1
            i = 0
            j = 0
            n = len(a1)
            while i < n and j < n:
                if a1[i] < a2[j]:
                    if smallestCounter == r:
                        return a1[i]
                    smallestCounter += 1
                    i += 1
                else:
                    if smallestCounter == r:
                        return a2[j]
                    smallestCounter += 1
                    j += 1

            if i == n:
                while j < n:
                    if smallestCounter == r:
                        return a2[j]
                    smallestCounter += 1
                    j += 1
            else:
                while i < n:
                    if smallestCounter == r:
                        return a1[i]
                    smallestCounter += 1
                    i += 1
      \end{lstlisting}
      The running time of this algorithm is $O(n)$

    % Question 2
    \item
      \begin{enumerate}
        \item
          Each partition is at least $\lfloor{\sqrt{n}}\rfloor$ in length.
        \item
          \[ \Theta ((2\lfloor{\sqrt{n}}\rfloor)^2) = \Theta (4n) = \Theta (n)\]
        \item
          \[ T(n) = T(\lfloor{\sqrt{n}}\rfloor) + T(n - \lfloor{\sqrt{n}}\rfloor) + O(n) \]
      \end{enumerate}

    \item
      \begin{enumerate}
        \item
          \[ (B_1, B_2), (B_3, B_4), (B_5, B_6), (B_7, B_8) \]
          \[ (B_2, B_4), (B_5, B_8) \]
          \[ (B_2, B_8) \]
          \[ \boxed{B_2} \]
        \item
          For the first round of weighting, the algorithm does $\frac{n}{2}$ weighings, and in the
          next round $\frac{n}{4}$ weighings. The number of weighing is halved as the rounds progress.
          As a result, the total number of rounds done is $\log _2 n$.
          The total number of weighings can be represented by a geometric series with first term $\frac{n}{2}$
          and ratio $\frac{1}{2}$.
          \[ S = \frac{n}{2}*\frac{\frac{1}{2}^{\log _2 n} - 1}{\frac{1}{2} - 1}
               = \frac{n}{2}*\frac{n^{-1} - 1}{-\frac{1}{2}} = \frac{n}{2}*(-2\frac{1}{n}+2) \]
          \[ S = n - 1 \]
          Therefore the number of weighings will not exceed n.
        \item
          The second heaviest ball must have been compared with the heaviest weight at some point during the
          algorithm. Checking all of weights that were compared with the heaviest weight requires an additional $\log _2 n$ of weighings.
        \item
          The elements that were compared with the heaviest weight $B_2$ are ${B_1, B_4, B_8}$. We must find
          the maximum in this dataset. So we would compare $B_1$ to $B_4$, $B_4$ to $B_8$, and $B_1$ to $B_8$
          to find that $B_8$ is the second largest weight.
      \end{enumerate}

    \item
      The algorithm randomly chooses a pivot, and partitions the array around that pivot. Then recursively call
      the partition with the larger cumulative weight (including the pivot). However, it also increases the pivot
      weight by the weight of the eliminated partition to account for the loss of weight. The time complexity is $O(n)$.
  \end{enumerate}
\end{document}
