\documentclass[11pt, letterpaper]{article}
\usepackage[top=80pt,bottom=80pt,left=60pt,right=60pt]{geometry}
\usepackage{listings}
\usepackage{amsmath}
\usepackage{titling}
\newcommand{\subtitle}[1]{
  \posttitle{
    \par\end{center}
    \begin{center}\large#1\end{center}
    \vskip0.5em}
}

\begin{document}
  \title{CSCI 3104 Assignment 9}
    \subtitle{10:00 - 10:50 Wanshan}
    \date{April 28 2016}
    \author{Jackson Chen}
    \maketitle

  \begin{enumerate}
    % Question 1
    \item
      \begin{enumerate}
        \item
            Start at the root node, then progress through the graph. Once the algorithm reaches the end of a branch,
            then backtrack and then proceed along the first edge path that hasn't been used yet. If the algorithm is going
            along a new path but reaches a node that has already been visited, then it is not a spanning tree. If at the end of this
            all of the nodes were visited and none were visited twice (except during backtracking), then it is a spanning tree.
        \item
          The point of a minimum spanning tree is to have the smallest sum of weighted paths connecting the tree. If there is an edge
          in the tree that is greater than an edge outside of the tree connecting two points, then during the construction of the tree,
          a safe edge was not added since during the step that added an edge of weight greater than $w$, the light edge (which is the
          edge of weight $w$) was not added. Therefore $T$ is not a MST of $G$.
        \item
          None of the MST Path Greatest Weights are greater than $e: (u, v)$. Therefore $T$ is a MST.
          \begin{table}[h]
          \centering
          \begin{tabular}{llll}
          \textbf{Edge}     & $e: (u, v)$     & MST Path        & MST Path Greatest Weight   \\
          \textbf{3, 4}     & 1.0               & 3-2-7-4       & 0.8                \\
          \textbf{5, 7}     & 1.2               & 5-4-7         & 1.1                \\
          \textbf{6, 7}     & 1.1               & 6-1-2-7       & 0.6                \\
          \textbf{6, 8}     & 0.8               & 6-1-8         & 0.5       \\
          \textbf{8, 9}     & 0.8               & 8-1-6-9       & 0.7         \\
          \textbf{9, 10}    & 0.7               & 9-6-1-2-7-10  & 0.7              \\
          \end{tabular}
          \end{table}
        \item
          $e_{min}$ would be the smallest $n$ edges in the graph up to the point where two weight edges are identical. Since
          in part B, the construction of an MST consists of adding safe edges, which means all the light edges (the minimum edges
          between $S$ and $V$) are appended to the MST. Therefore the smallest $n$ edges (which is $e_{min}$) must be added to the MST.
        \item
          Prim's algorithm only adds safe nodes (which involves taking light edges, edges that satisfy certain requirements
          as well as takes the lowest weight edges that cross a cut). When node $u$ is the starting node, the algorithm will take
          the edge $(u, v)$ because it is a safe edge (it crosses a cut that respects T and is a light edge).
        \item
          \begin{itemize}
            \item
              The same tree except with the edge $9-10$ instead of $7-10$.
            \item
              The same tree except with the edge $9-10$ instead of $6-9$.
          \end{itemize}
      \end{enumerate}
    % Question 2
    \item
      \begin{enumerate}
        \item
          There are 25 decision variables from $x_{11}$ to $x_{55}$, represented in the general form $x_{ij}$. When person $i$ does
          task $j$, then $x_{ij}$ is 1, otherwise it is 0.
        \item
          $x_{15} + x_{25} + x_{35} + x_{45} + x_{55} = 1$
        \item
          Assuming $T = 35$. \\
          $x_{41}*T_1 + x_{42}*T_2 + x_{43}*T_3 + x_{44}*T_4 + x_{45}*T_5 = 14$
        \item
          $A_3 = x_{31}*P_{3}[0] + x_{31}*P_{3}[1] + x_{31}*P_{3}[2] + x_{31}*P_{3}[3] + x_{31}*P_{3}[4]$
        \item
          \begin{lstlisting}
            import math

            t = [5, 10, 2, 8, 10]
            p = [[2, 1, 3, 4, 5], [3, 2, 4, 1, 5], [4, 2, 3, 5, 1], [1, 2, 3, 5, 4], [3, 4, 2, 1, 5]]

            temp = []
            for i in range(24):
                temp.append(0)

            mina = 100000
            for i in range(0, 21):
                for j in range(i+1, 22):
                    for k in range(j+1, 23):
                        for l in range(k+1, 24):
                            for m in range(l+1, 25):
                                a = p[int(math.floor(i/5))][i % 5] +
                                    p[int(math.floor(j/5))][j % 5] +
                                    p[int(math.floor(k/5))][k % 5] +
                                    p[int(math.floor(l/5))][l % 5] +
                                    p[int(math.floor(m/5))][m % 5]
                                if a < mina:
                                    mina = a
            print mina
          \end{lstlisting}
          Answer \boxed{5}
      \end{enumerate}
  \end{enumerate}
\end{document}

% \begin{lstlisting}
